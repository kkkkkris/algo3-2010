La estructura del grafo está dada por un arreglo de nodos, donde cada posición es el $id-1$ del nodo que contiene. A su vez los nodos tienen su propio $id$ y una listas con los de sus vecinos. Cabe aclarar que $1 \leq id \leq n$ donde $n$ es la cantidad de nodos del grafo.

A la hora de realizar las mejoras tuvimos que agregar más información a la estructura. Al grafo le añadimos $m$, es decir la cantidad de aristas. Además incorporamos a cada nodo su orden de presedencia para ser pivote (siempre elige al que tiene el menor). Esto permite cambiar facilmente el criterio de elección y además mejora la performance en nuestro caso, ya que le damos el orden al crear el grafo de acuerdo a su cantidad de vecinos y nunca más nos tenemos que preocupar por recorrer el grafo para obtener esta información. 

Para implementar el algoritmo de Bron–Kerbosch decidimos no utilizar conjuntos, sino listas. La razón es que nos permite realizar las operaciones de conjunto a nosotros según la conveniencia del caso.
