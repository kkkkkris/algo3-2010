El ejercicio número dos presentado en este Trabajo Práctico plantea que dado un grupo de amigas, entre las cuales pueden ser o no, amigas (una relacion simétrica, pero no transitiva). Decidir si este grupo de alumnas pueden formar una ronda de amigas que las contenga a todas. Luego, traduciendo las alumnas y su relación como amigas a un grafo, donde cada alumna es un nodo y la relación un vertice que une esos nodos. Se puede pensar al problema como encontrar un camino que comience en un vértice en particular y recorriendo a travéz de las aristas todas las amigas exactamente una vez y regrese al punto de partida. Encontramos que éste último concepto es descripto como un Ciclo Hamiltoniano (Ver http://es.wikipedia.org/wiki/Camino\_hamiltoniano).

Leyendo un poco sobre esto, encontramos que encontrar si un grafo genérico (esto significa un grafo del que no se tienen hipótesis previas) es Hamiltoniano es NP-Completo. Por lo que no es posible dar una solución de orden polinomial a este problema. Al escuchar esto uno podría entristecerse un poco.

Sin embargo, si existen algunas caracterizaciones de los grafos hamiltonianos tal como el Teorema de Ore y el Teorema de Dirac, los cuales mostraremos más adelante, que permiten definir si un grafo es Hamiltoniano (permiten afirmar, no negar) con orden polinomial.
