En éste ejercicio contamos con el problema de la subsecuencia unimodal más larga, dada una secuencia, debemos hallar la cantidad mínima de elementos a eliminar tal que nos quede como resultante la subsecuencia (o una de ellas si hay más de una) unimodal máxima.

Una secuencia es unimodal si es creciente hasta cierto elemento y decreciente a partir del mismo.
El problema debe ser resuelto en una complejidad menor que O(n$^{3}$).

Consideramos abordar el problema dividiéndolo en dos subproblemas, el de hallar la longitud de la subsecuencia creciente máxima para una instancia de la secuencia (o sea para S[1..i]) y lo mismo para la subsecuencia decreciente máxima (para S[i..n]).Luego calculamos la longitud de la subsecuencia unimodal máxima que tiene como pico máximo a un elemento particular de la secuencia usando las longitudes de las subsecuencias máximas creciente y decreciente correspondientes.

Para el cálculo de la complejidad se eligió el modelo de costo uniforme, dado que no hay aclaraciones respecto de que el tamaño de la entrada pudiera no ser acotado.

Se realizaron las experimentaciones midiendo el tiempo de procesamiento del algoritmo excluyendo los tiempos de parseo de archivo y escritura de resultados.

Finalmente se contrastaron los resultados obtenidos ,con los calculados teoricamente. 