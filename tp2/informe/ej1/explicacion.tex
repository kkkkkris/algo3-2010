Como contamos anteriormente, proponemos encontrar la subsecuencia unimodal máxima a partir del conjunto de todas las secuencias unimodales maximales. Para generar tales subsecuencias hacemos la unión de dos subsecuencias de la secuencia original, la primera creciente y la segunda decreciente, de modo tal que cumplan ser máximas y el primer elemento de la secuencia decreciente sea el ultimo de la secuencia creciente. Para ello generamos todas las subsecuencias decrecientes y crecientes máximas que cumplen que comienzan y terminan en un determinado elemento.

\subsubsection{Correctitud}
Sea $s$ una secuencia de enteros, definimos $l_i$ como la subsecuencia creciente máxima de $s_[1..i]$ que contiene al elemento $s_i$ ($l_i \in
  \{l_j ++ [s_i] / \exists j \in \mathbb{N}, j < i \land s_j < s_i\}$ )y la función maxCreciente como la que devuelve el tamaño de dicha $l_i$ para un $i$ determinado definida asi:\\
\\
    \centerline{$ \displaystyle maxCreciente(num\ i) = \max_{j \in A_i} (maxCreciente(j)) + 1$} 
    \centerline{tal que $A_i = \{ j \in \mathbb{N} / j < i \land s_j < s_i \}$}

Demostremos por inducción en i que $maxCreciente(i) = largo(l_i)$:

Caso base\\
i = 1 \\
\centerline{$\displaystyle maxCreciente(1) = \max_{j \in A_1} (maxCreciente(j)) + 1$}
\centerline{$\displaystyle maxCreciente(1) = \max_{j \in \emptyset} (maxCreciente(j)) + 1$}
\centerline{$\displaystyle maxCreciente(1) = 1 = largo(s) = largo(l_1)$} 
Como queriamos ver.\\
\\
Paso Inductivo\\
Supongo que vale $\forall j \in \mathbb{N} / j < i$, quiero ver que para $i$. $(i > 1)$ vale:\\
\\
$\displaystyle maxCreciente(i) = \max_{j \in A_i} \{maxCreciente(j)\} + 1 \ \ \ (1)$\\

Notemos que $ A_i \subset H = \{ j \in \mathbb{N} / j < i\} $ por lo que si vale para los elementos de $H$, tambien para los de $A_i$\\
Por hipótesis inductiva vale que:\\

$\displaystyle \forall j \in A_i , maxCreciente(j) = largo(l_j) \\
\\
 \Rightarrow \max_{j \in A_i}({maxCreciente(j)}) = \max_{j \in A_i}({largo(l_j)}) \\
 \\
 \Rightarrow \max_{j \in A_i}({maxCreciente(j) + 1}) = \max_{j \in A_i}({largo(l_j) + 1})$ \\
 \\
Como $s_i  \notin l_j$
\\
$\displaystyle \Rightarrow \max_{j \in A_i}({maxCreciente(j) + 1}) = \max_{j \in A_i}({largo(l_j ++ [s_i])}) \ \ \ (2)$ \\
\\
Como $\forall j \in A_i,\ s_i > s_j$ entonces $l_j ++ [s_i]$  sigue siendo una secuencia creciente.\\
\\
$\displaystyle \Rightarrow \max_{j \in A_i}({largo(l_j ++ [s_i])}) = largo(l_i)\ \ \ (3) $ por propiedad descripta en el enunciado. \\ 
\\
Por (1), (2) y (3)      
$\displaystyle \Rightarrow maxCreciente(i) = \max_{j \in A_i}({maxCreciente(j) + 1}) = \max_{j \in A_i}({largo(l_j ++ [s_i])})$ \\
\\
$\displaystyle \therefore maxCreciente(i) = largo(l_i) \blacksquare$ 

Cabe destacar que la misma demostración vale para la subsecuencia decreciente máxima,definiendo $l_i$ como la subsecuencia decreciente máxima de $s_[i..n]$ que contiene al elemento $s_i$ ($l_i \in
  \{ [s_i]++l_j  / \exists j \in \mathbb{N}, j > i \land s_j < s_i\}$ )y la función maxDecreciente como la que devuelve el tamaño de dicha $l_i$ para un $i$ determinado.

Una vez demostrado esto, mostremos que hallando la mejor combinación entre subsecuencia creciente y subsecuencia decreciente, obtenemos la subsecuencia unimodal óptima.\\
Llamemos $SC_i$ a una subsecuencia creciente maxima para S[1..i] que incluye al elemento $S_i$ y $SD_i$ a una subsecuencia decreciente maxima para $S[i..n]$ que incluye al elemento $S_i$(puede haber más de una).\\
Sea el "pico máximo" de una secuencia unimodal el elemento máximo de la misma y llamemos $U_i$ a una subsecuencia unimodal que tiene al elemento $S_i$ como su pico máximo.\\
Construimos $U_i$ formada por una combinación de $SC_i$ + fin($SD_i$),donde fin(s) devuelve s menos su primer elemento, ya que el último elemento de $SC_i$ y el primero de $SD_i$ son el mismo, en especial es el pico máximo. Entonces podemos ver que $U_i$ es maximal, no tengo forma de agregarle elementos (tanto de su parte izquierda como de su parte derecha), tal que siga siendo unimodal. Pues si hubiera forma, $SC_i$ y $SD_i$ no seriam máximas.\\
Luego tomando el conjunto $M = \{U_i / i \in \mathbb{N} \land i \le largo(s)\}$, habremos obtenido el conjunto de todas las posibles subsecuencias unimodales máximales (toda secuencia tiene al menos un elemento). De este conjunto $M$ tomamos la $U_i$ que sea máxima en largo y habremos obtenido la subsecuencia unimodal máxima. Con lo cual queda demostrado que es correcta la concatenación de subsecuencias crecientes y decrecientes máximas(o mejor dicho la suma de sus longitudes) para formar la unimodal.
\subsubsection{Complejidad}
Para poder resolver el problema se tomo como estructura una matriz de 3 x n que denominamos Tabla en el algoritmo para mayor claridad.\\
Para mejor comprensión del pseudocódigo indexaremos la tabla como tabla[i][number/up/down] donde :
\begin{itemize}
\item La columna \textbf{number} guarda el elemento i de la secuencia (un entero).
\item La columna \textbf{up} guarda el tamaño de máxima subsecuencia creciente desde 1 hasta i que incluye al elemento i.
\item La columna \textbf{down} guarda el tamaño de máxima subsecuencia decreciente desde i hasta n que incluye al elemento i. 
\item tabla[0][number] por convención contiene el largo de la secuencia.
\item tabla[0][up] y tabla[0][down] contienen 0.
\end{itemize}

\texttt{Pseudocódigo}

\textit{Función principal que calcula la cantidad mínima de elementos de una secuencia que hay que eliminar para que la subsecuencia resultante sea unimodal.}
\begin{verbatim}
    calcularMin(Tabla t){
        para cada posicion i de la secuencia
            tabla[i][up] = maxSubsecuenciaCreciente(t,i)
        para cada posicion i de la secuencia (se recorre en orden inverso)
            tabla[i][down] = maxSubsecuenciaDecreciente(t,i)
        /* obtenemos el tamaño de la maxima subsecuencia unimodal*/   
        max = unimodalMax(t)     
        devolver n-max
}
\end{verbatim}

\textit{Función que devuelve la maxima subsecuencia creciene que contiene al elemento $i$ de $t$.\\
Notemos que maxSubsecuenciaDecreciente es anàloga a maxSubsecuenciaCreciente.}

\begin{verbatim}                   
    maxSubsecuenciaCreciente(Tabla tabla,num i)    
        para j=i-1,j>0,j--                                   O(i-1)
            si tabla[j][number] < tabla[i][number]
                m = maximo(tabla[i][up], tabla[j][up]) 	
        devolver m+1;
\end{verbatim}


\textit{La siguiente función una vez obtenidos los tamaños de máxima subsecuencia creciente y máxima subsecuencia decreciente busca la combinación de ambas que conforman la máxima subsecuencia unimodal y devuelve el tamaño de la misma.
Lo que hace es recorrer la tabla y para cada fila de la misma (con sus tres columnas number ,up y down) suma up y down y chequea si es mayor al max acumulador.
Finalmente ,resta uno al max obtenido ya que sumamos dos veces el "pico" de la subecuencia unimodal máxima ,ya que tanto up como down incluían a ese elemento.}

\begin{verbatim}
    unimodalMax(tabla t){
        res = 0 ;
        para cada posicion k de la tabla                       O(n)
            res = max(t[k][up]+t[k][down], res)
        devolver max-1
}
\end{verbatim}


Analicemos las complejidades de este algoritmo:

\begin{verbatim}
    calcularMin(Tabla t){
        para cada posicion i de la secuencia                     O(n^2)
            tabla[i][up] = maxSubsecuenciaCreciente(t,i)
        /* (se recorre en orden inverso) */
        para cada posicion i de la secuencia                     O(n^2)  
            tabla[i][down] = maxSubsecuenciaDecreciente(t,i)
        /* obtenemos el tamaño de la maxima subsecuencia unimodal*/   
        max = unimodalMax(t)                                      O(n)  
        devolver n-max
}                                                                ____
                                                          total  O(n^2)
\end{verbatim}

Como se observa en el pseudocódigo el algoritmo se compone de dos algoritmos de búsqueda de subsecuencia creciente y decreciente más larga. Ambos funcionan de forma análoga, por lo que de ahora en más vamos a analizar solo el algoritmo de búsqueda de secuencia máxima creciente y concluir los mismos resultados para el otro.\\
Analizando la complejidad del algoritmo podemos ver dos O($n^{2}$) de las funciones maxSubsecuenciaCreciente y maxSubsecuenciaDecreciente, éstos provienen de la siguiente sumatoria:

el primer ciclo hace\\
$\displaystyle\sum_{i=1}^n maxSubsecuenciaCreciente(t, i) = \sum_{i=1}^{n} i-1  = \frac{n(n-1)}{2} = O(n^{2})$
Luego O($n^{2}$)+O($n^{2}$)+O(n)=O($n^{2}$)              
Mostramos entonces que la complejidad de nuestro algoritmo es O($n^{2}$).
