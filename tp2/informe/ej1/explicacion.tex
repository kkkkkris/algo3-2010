La estructura soporte utilizada en el algoritmo es una matriz de 3 x n que denominamos Tabla en el algoritmo para mayor claridad.
\newline
La Tabla almacena para cada fila i: en una columna el elemento  i de la secuencia, en otra el tamaño de la máxima subsecuencia creciente desde 1 hasta i que incluye al elemento i de la secuencia, y en la tercera , el tamaño de la máxima subsecuencia decreciente desde  i hasta n que incluye al elemento i de la secuencia.
\newline
Para mejor comprensión del pseudocódigo indexaremos la tabla como tabla[i][number/up/down] donde la columna number  tiene el elemento en si,la columna up el tamaño de máxima subsecuencia creciente y la columna down el tamaño de máxima subsecuencia decreciente.
En tabla[0][number] se guardara el tamaño de la secuencia y lo notamos como n en el pseudocódigo.
\newline
Pseudocódigo

$/*$función que llena una fila de la tabla con los tamaños de máxima subsecuencia creciente desde 1 hasta i que incluyen el elemento i  $/*$

\begin{verbatim}                        
llenar_tablaup(tabla,i)           
   para j=i-1,j>0,j--                   O(i-1)   
     m= max_predecesor?(tabla,i,m) 	
   devolver m+1;
\end{verbatim}    
        

$/*$función que llena una fila de la tabla con los tamaños de máxima subsecuencia decreciente desde i hasta n que incluyen el elemento i $/*$ 
       
\begin{verbatim}                        
  llenar_tabladown(tabla,i)             
     para j=i+1,j<n+1,j++                 O(n-i) 
       m= max_predecesor?(tabla,i,m) 	 
     devolver m+1;
\end{verbatim} 

max\_predecesor?(tabla,i,m) son comparaciones O(1) que devuelven el maximo predecesor hasta el elemento i. 

$/*$La siguiente función una vez obtenidos los tamaños de máxima subsecuencia creciente y máxima subsecuencia decreciente realiza la búsqueda de la combinación de ambas que conforman la máxima subsecuencia unimodal y devuelve el tamaño de la misma.
Lo que hace es recorrer la tabla y para cada fila de la misma (con sus tres columnas number ,up y down) suma up y down y chequea si es mayor al max acumulador.
Finalmente ,resta uno al max obtenido ya que sumamos dos veces el "pico" de la subecuencia unimodal máxima ,ya que tanto up como down incluían a ese elemento. 

\begin{verbatim}
  unimodalMax(tabla t){
      int max,aux; 	
      max= 0 ;
      para k=1,k<n,k++                    O(n)
	    aux=t[k][up]+t[k][down]
	    si aux>max 
	       max=aux
	    finsi 
	  finpara
	  max-- 
      devolver max
  }
\end{verbatim}


$/*$función principal que calcula la cantidad mínima de elementos de una secuencia que hay que eliminar para que la subsecuencia resultante sea unimodal.$*/$

\begin{verbatim}
 calcularMin(Table t){
    para i=1,i<n+1                   O(n2)
       llenar_tablaup(t,i) 
    para i=n,i>0                     O(n2)
       llenar_tabladown(t,i)
    
    max= unimodalMax(t)              O(n)
    devolver n-max   
}                                    ____
                            total:   O(n2)
\end{verbatim}                       

Los dos O($n^{2}$) de los ciclos con llenar\_tablaup y llenar\_tabladown provienen de la siguientes sumatorias:

el primer ciclo hace

$\sum_{i=1}^{i=n}$ llenar\_tablaup(t,i)= $\sum_{i=1}^{i=n} i-1 $ = n(n-1)$/$2 = O($n^{2}$)

mientras que el segundo

$\sum_{i=n}^{i=1}$ llenar\_tabladown(t,i)= $\sum_{i=1}^{i=n}(n-i)$ = $\sum_{i=1}^{i=n}(i-1)$ = (n-1)n$/$2 = O($n^{2}$)
                       
Mostramos entonces que la complejidad de nuestro algoritmo es O($n^{2}$).

Como se observa en el pseudocódigo el algoritmo se compone de dos algoritmos de búsqueda de subsecuencia creciente y decreciente más larga.

Llamando S a la secuencia,lu(long up)a la secuencia de longitudes de subsecuencias máximas crecientes  y ld(long down) a la secuencia de longitudes de subsecuencias máximas decrecientes,la relación de recurrencia de los algoritmos es :

lu$_0$=0;

lu$_i$ =$\max_{j<i}$ lu$_j$+ 1  tal que S$_j$ $<$ S$_i$  

y la solución la da $\max_{1\leq i\leq n}$ lu$_i$

ld$_0$=0;

ld$_i$ =$\max_{i<j}$ l$_j$+ 1  tal que S$_j$ $<$ S$_i$ 

y la solución la da $\max_{1\leq i\leq n}$ ld$_i$


El principio de optimalidad vale para ambos y se podría enunciar así : 

Si una subsecuencia es óptima(mas larga), no puede haber elementos entre los elementos de la subsecuencia que formen otra subsecuencia mejor que las subsecuencias que forman los elementos de la óptima, en otras palabras, las subsecuencias formadas por los elementos de la óptima, son óptimas para esa longitud.
Luego hallando la mejor combinación entre subsecuencia creciente y subsecuencia decreciente, obtenemos la subsecuencia unimodal óptima. 

%Es decir maxSecUnimodal = max(lu$_i$+ld$_i$)-1\hspace{0.1cm} $\forall$ i,j, $1\leq i<j\leq n$,S${_i} \not =$x${_j}$ 

Llamemos SC${_i}$ a una subsecuencia creciente maxima para S[1..i] que incluye al elemento S$_i$ y SD${_i}$ a una subsecuencia decreciente maxima para S[i..n] que incluye al elemento S${_i}$.(puede haber más de una máxima)

Llamemos U${_i}$ a una subsecuencia unimodal máxima que tiene al elemento S${_i}$ como su pico máximo.
Tenemos entonces que necesariamente U${_i}$ está formada por una combinación de SC${_i}$ + fin(SD${_i}$) (donde fin(s) devuelve s menos su primer elemento), ya que el primer elemento de SC${_i}$ y el último de SD${_i}$ son el mismo y son el pico máximo de la subsecuencia unimodal U${_i}$.

Y lo anterior se deduce de que si S{$_i$} es pico máximo, la máxima U${_i}$ estará formada por la subsecuencia creciente más larga hasta el pico y la decreciente mas larga desde el pico. 

Luego tendremos la subsecuencia unimodal máxima en $\max_{1\leq i\leq n}$ U${_i}$, y estamos considerando todas las posibles subsecuencias unimodales máximas dado que toda secuencia unimodal tiene un "pico" máximo.
Con lo cual es correcta la concatenación de ambas subsecuencias(o la suma de longitudes según lo pedido en el ejercicio) para formar la unimodal. 
