El siguiente ejercicio tiene por objetivo decidir si, dado un conjunto de esquinas y calles que las unen, es posible asignar un sentido a cada calle de forma tal 
de poder viajar de cualquier esquina hacia cualquier otra. \\
La situación es fácilmente modelable con un \textit{grafo}, donde cada esquina es representada como un \textit{vértice}, y cada calle como una \textit{arista} que los une. El problema entonces se transforma en decidir si es posible asignar sentidos a las aristas del grafo tal que al finalizar de asignar un sentido a todas ellas, el grafo resultante, al que llamamos \textit{grafo dirigido} o \textit{digrafo}, resulta \textit{fuertemente conexo}. \\
A continuación daremos algunas definiciones y demostraremos algunas propiedades que nos permitirán desarrollar un algorítmo para responder a la pregunta de la asignación de sentidos.