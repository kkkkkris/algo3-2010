
En este ejercicio se pide que, dados ciertos pasillos que unen una determinada cantidad de habitaciones, y sabiendo que algunas contienen puertas y otras llaves para abrirlas, ver si es posible llegar de la primera a la última habitación (es decir de la \textit{1} a la \textit{n}). \\

Este problema claramente se puede modelar con un grafo, donde cada habitación es un nodo que puede contener una puerta, una llave (la cual abre una única puerta) o nada, y cada pasillo es una arista. \\

Luego, resolver el ejercicio pedido es equivalente a ver si se puede llegar al nodo \textit{n} (donde \textit{n} es la cantidad de nodos) partiendo desde el \textit{1}. Esto debe hacerse teniendo en cuenta que es necesario pasar por algunos nodos como requisito para pasar por otros (hay que conseguir la llave correspondiente a cada puerta).
En las siguientes secciones explicaremos en detalle el modelo y el algoritmo utilizado para resolver el problema, y también analizaremos la complejidad del mismo. \\