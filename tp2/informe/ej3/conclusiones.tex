La conclusión que realizamos al terminar de resolver el problema, es que al llevarlo a un nivel más abstracto es claro que se trata de recorrer un grafo con ciertos nodos prioritarios. Con lo cual, nos parece lógico, y una buena solución, haber utilizado \textit{BFS} con un criterio de parada más fuerte y agregando la prioridad. \\

Finalmente, pudimos comprobar que nuestras consideraciones teóricas se cumplen en la práctica al realizar las pruebas con nuestro generador de grafos para el peor caso. Es decir que hay una relación lineal entre el tamaño del entrada $n+m+p$ y el tiempo de ejecución del algoritmo, como se puede observar en el gráfico de la sección pruebas. Por lo que creemos que un algoritmo con complejidad $O(n+m+p)$ en el peor caso es muy bueno.\\
