El ejercicio compila con el comando \texttt{make} .\\
El algoritmo recibe como parámetros de entrada tres nombres de archivo \emph{'entrada'},\emph{'salida'} y \emph{'tiempos'}, y se ejecuta \texttt{./secuencias entrada.in salida.out tiempos.times}.\\
El archivo \emph{'entrada'} (.in) contiene una lista de secuencias de enteros cada secuencia en una línea precedida de su longitud y el fin de la lista lo determina una línea con -1.\\
El archivo \emph{'salida'} (.out)lo genera el algoritmo con los resultados de cada secuencia(cantidad de mínima de elementos a eliminar de la secuencia tal que la subsecuencia resultante sea unimodal).\\
Por último el archivo \emph{'tiempos'} (.times)  también generado por el algoritmo, contendrá una tabla con el tamaño de cada secuencia y tiempo de ejecución correspondiente, para cada una.\\
La generación de secuencias 'entrada' se realizó con un generador de secuencias(véase\\ \texttt{generador\_de\_secuencias/generador\_secuencias.cpp}), éste compila con el comando make y se ejecuta con 
\texttt{./secuencias modo archivo\_entrada.in  min max escala}, donde modo debe ser r=random(no implementamos otro), archivo entrada.in es el nombre del archivo donde va a volcar la lista de secuencias generada, min es el valor mínimo de tamaño de secuencia, max es el máximo y escala la diferencia de tamaño entre secuencias conjuntas a generar en la lista.\\
El generador genera secuencias de Naturales, no de Enteros pero consideramos que a efectos experimentales el comportamiento es el mismo para números negativos, de todas formas se realizaron tests para los mismos.
Agregar el parámetro escala tanto como min y máx fue necesario, ya que en un principio, al probar con secuencias de menos de 20 elementos, e incrementando a escalas pequeñas no se observaba claramente la relación cuadratica de los tiempos, luego encontramos un razonable set de datos para las experimentaciones definitivas, listas de secuencias de 20 a 800 elementos incrementando de a 10, 20 y 40.  
