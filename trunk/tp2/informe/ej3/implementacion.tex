Para compilar el programa, basta con ejecutar el makefile adjunto. Para ejecutarlo hay que pasar al menos dos parámetros, ambos nombres de archivos. El primero es el archivo de entrada desde donde el programa leerá los datos de cada prisión con el formato especificado en el enunciado del tp. El segundo es el archivo de salida donde se escribirá la respuesta, también siguiendo las pautas del enunciado del tp.
Además hay un tercer parámetro opcional que permite dar un archivo más en donde se guardaran, en una primera columna, el tamaño de cada entrada ($n+m+p$) y, en otra columna, los tiempos de ejecución de cada instancia.