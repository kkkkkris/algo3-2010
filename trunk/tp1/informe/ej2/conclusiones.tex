En vistas de los resultados, dado que no pudimos generar lotes de peores casos, es imposible apreciar la complejidad calculada con datos de pruebas. Sin embargo podemos observar que a medida que los grafos son mas ``densos'' en el sentido de tener más vecinos cada nodo, en general tienden a caer bajo alguno de los teoremas de existencia del ciclo de Hamilton, con lo que los tiempos son relativamente bajos. Por el contrario, para grafos poco ``densos'' es mucho menos probable encontrar un ciclo de Hamilton, y para los valores de n cada vez más altos, el tiempo crece enormemente.
Como conclusión final, diremos que si bien es un problema que computacionalmente no esta bien resuelto, existen algorítmos mejores que el implementado en este tp utilizando distintas técnicas de programación (dynamic programming) que obtienen complejidades mejores.