Como solución principal al problema de decidir si existe ronda representamos a las alumnas como un grafo, e hicimos un algoritmo que comenzando en algun vértice fijo recorre todos los caminos que se puedan formar sin pasar dos veces por un mismo vértice. De forma tal que si encuentra un camino que lo lleva al principio habiendo pasado por todos los vértices, luego para esa instancia existe una ronda y efectivamente es esa. En caso de haber recorrido todos los caminos posibles mencionados devuelve que no existe forma de hacer tal ronda.

Explicado de una forma mas procedimental, se comienza por un nodo, en nuestro caso la primer alumna. Entonces, se analiza para cada vértice $v$ vecino (que es amiga) del actual, que no haya sido visitado en el camino actual, si existe un camino que pasando por $v$ llegue al inicial pasando por el resto de los vértices no visitados en este camino. Cuando en un vértice no quedan mas vértices vecinos que visitar, esto significa que todos estos vértices fueron probados como camino y/o visitados anteriormente por este mismo camino, entonces se vuelve al vértice desde el que se visitó al actual. Y se sigue analizando el resto de vértices vecinos a este vértice anterior, siempre tomando la misma acción en caso de que se acaben los vecinos disponibles para analizar.

\newpage
A continuación un seudocódigo de nuestro algoritmo en su forma recursiva:

\begin{algorithm}
{esHamiltoniano}{\param{}{alumnas}{Grafo}}{bool}
	\VAR{primera:Nodo} \\
	primera \= alumnas.primera();\\
	res \= hayCamino(primera,\ primera,\ <>, alumnas.tamaño());\\
	return\ res;
\end{algorithm}

\begin{algorithm}
{hayCamino}{\param{}{actual}{Nodo},\ \param{}{inicial}{Nodo},\ \param{}{recorrido}{list},\ \param{}{tamaño}{int}}{bool}
{	recorrido.agregar(actual);\\
	\begin{IF}{encontreCamino(actual, inicial, recorrido, tamaño)}
		return\ true;
	\end{IF}\\
	//$itero sobre los vecinos del nodo actual$\\
    \begin{FOREACH}{\param{}{amiga}{Nodo}}{actual.vecinas()}{
        \begin{IF}{!\ visitado(recorrido, amiga) }
			\begin{IF}{hayCamino(amiga, inicial, recorrido, tamaño)}
			    return\ true;\\
			 \end{IF}
		\end{IF}}
    \end{FOREACH}
	\hspace{-3.4cm}recorrido.quitar(actual);\\
	return\ false;}
\end{algorithm}

\begin{algorithm}
{encontreCamino}{\param{}{actual}{Nodo},\ \param{}{inicial}{Nodo},\ \param{}{recorrido}{list},\ \param{}{tamaño}{int}}{bool}
    return\ recorrido.largo == tamaño\ \&\&\ inicial.vecinas().contiene(actual);
\end{algorithm}

\begin{algorithm}
{visitado}{\param{}{recorrido}{list},\ \param{}{amiga}{Nodo}}{bool}
    return\ recorrido.contiene(amiga);
\end{algorithm}


    
Ya que nuestro algoritmo no repite vértices no puede ocurrir que un camino "se pierda", con esto queremos decir que entre en un ciclo del que no pueda salir. Tampoco puede ocurrir que el algoritmo repita caminos, ya que los va probando de forma ordenada descartando sin posibilidad de reuso los caminos inservibles. Por estas ultimas dos cosas podemos asegurar que el algoritmo termina, ya que dentro de un grafo solo existe una cantidad finita de caminos que no repiten elementos.

El algoritmo recorre todos los caminos que se pueden formar desde un vértice sin repetir vértices. Esto significa que si no se terminara cuando se encuentra el ciclo hamiltoniano, terminaria generando todos los caminos posibles que tienen todos sus vértices distintos. Por generar todos los caminos que tienen sus vértices distintos y por terminar, se puede asegurar que el algoritmo encontrará si existe o no un ciclo hamiltoniano, ergo el algoritmo es correcto.

Para medir la complejidad de este algoritmo elejimos el modelo uniforme, ya que el costo de las operaciones elementales no varia cuando cambia el tamaño de la entrada. Tomamos como función para medir el tamaño de la entrada la cantidad de amigas.


