Como era de esperar, el gráfico nos describe una función aproximadamente lineal, observamos que en el que consideramos el peor caso, la pendiente de la 'cuasi recta' es mas pronunciada que en los casos randoms o casos promedio,se puede observar que alcanza valores del orden de $10^{5}$ para planillas de 4000 datos, contra los de orden de $10^{4}$ de los lotes randoms para el mismo tamaño de datos. 
\newline
Sin embargo a los efectos del cálculo de la complejidad, ésto no hace diferencia, ya que la diferencia está dada por una constante multiplicativa .
