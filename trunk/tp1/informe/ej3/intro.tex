En éste ejercicio contamos con el problema de la oficina de programadores, hay una lista con dos sublistas de horarios de entrada y salida de cada uno de los programadores que trabaja en una oficina , por cada entrada hay una salida respectiva y además las sublistas estan ordenadas en orden creciente de tiempo.
\newline
El problema pide obtener la cantidad máxima de programadores que se encuentran en simultáneo en la oficina en algún instante.
\newline
La solución tomada fue ir recorriendo simultáneamente las dos sublistas y a modo de mergeo ir comparando las horas de entrada y salida próximas, actualizando un contador que nos termina dando el resultado final.
\newline
Para el cálculo teórico de la complejidad se eligió el modelo de costo uniforme obteniendo como resultado una complejidad lineal del algoritmo en funcion del tamaño de la entrada.
\newline
Se realizaron las experimentaciones midiendo el tiempo de procesamiento del algoritmo excluyendo los tiempos de parseo de archivo y escritura de resultados.
Finalmente se contrastaron los resultados obtenidos ,con los calculados teoricamente. 