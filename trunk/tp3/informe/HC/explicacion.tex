Para lograr una solucion al problema de MAX\_CLIQUE para un grafo $G(V,E)$, implementamos una heurística constructiva y otra local. Tanto la heurística constructiva como la heurística local se basan en una idea de densidades dentro del grafo y consiste en orientar la búsqueda del clique máximo priorizando las zonas de mayor densidad. Para ello definimos una familia de funciones que llamaremos $delta: V \longrightarrow \mathbb R$, la cual intenta mostrar la densidad del grafo para cualquier $Nodo n \in V$.
Para la heurística constructiva definimos como función objetivo en primeria instancia maximizar el tamaño del clique, y en segundo maximizar el valor de delta para el nodo que se agrega. Mientras que para la heurística local, se maximizó primero el tamaño del clique, y luego la sumatoria de los deltas de los nodos pertenecientes al clique, lo que permitio intercambiar nodos del clique por otros con mayor $delta$ sin necesidad de achicar el clique.
Experimentamos con 2 medidas diferentes de densidad, definidas como $\delta$1 y $\delta$2, que son asignadas a cada nodo antes de comenzar el algoritmo.
Definimos $\displaystyle \delta$1 para el nodo i  como el promedio de los grados de todos sus vecinos y de el mismo ,

   $\displaystyle \delta1(i) = \frac{\sum_{j=1,j-vecino-i}^{n} grado(j) + grado(i)}{grado(i)+1} $

Por otro lado definimos $\displaystyle \delta$2 para el nodo i como la suma de los grados de todos sus vecinos y de el mismo. 
	
   $\displaystyle \delta2(i) = {\sum_{j=1,j-vecino-i}^{n} grado(j) + grado(i)}$

Como primera instancia pensamos $ \delta2(i)$ sin embargo consideramos que había casos en los que la suma de los grados de los nodos vecinos no nos orientaba bien hacia cliques máximos, podría caer a explorar nodos con muchos vecinos(tipo estrella) pero que no formaban cliques, por lo tanto surgió la idea del $ \delta1(i)$ con el que probamos en primer lugar.
Este tiene la particularidad de que incluye en el promedio el grado del nodo actual, con lo cual dados dos nodos con igual promedio entre sus vecinos (por ej uno con 1 solo vecino de grado 3 y otro con 10 vecinos de grado 3)prioriza al que tiene mayor grado (el de 10). 

%RESULTADOS %%%
Sin embargo luego de hacer corridas con ambos deltas, observamos que si bien salvábamos esos casos, había otros en los que era mejor utilizar el  $ \delta2$ ,tal como se puede observar en la tabla de resultados(ver dataset p-hat). 
Un caso patologico para el $\delta1$ puede ser uno en el que el delta de un nodo aumenta mucho debido a que tiene nodos vecinos con grados altos pero éste tiene grado chico y no forma clique, luego en ese caso conviene volcarse a elegir el nodo con $ \delta2$ mayor.
%%%%%%%%%%%%%%
\subsubsection{Complejidad}
Analizaremos las complejidades temporales usando el modelo uniforme de ambas heurísticas.

\textbf{Heurística Constructiva}\\
\\
Una vez calculadas las densidades de todos los nodos, tomamos el nodo de mayor densidad y empezamos la búsqueda del cliqué máximo que contiene a ese nodo,insertándolo en el conjunto solución Cq e insertando a todos sus vecinos en una cola de prioridad S(prioridad de máxima densidad).
La cola de prioridad S contendrá durante todo el algoritmo a todos los candidatos posibles a formar parte del cliqué obtenido hasta el momento Cq.
Una vez vacía S el algoritmo termina, en el peor caso el cliqué máximo es el grafo completo y todos los nodos son en algún momento insertados en S y luego quitados e insertados en Cq.

\begin{verbatim}
generarDensidad()										O(n*n)
	
merge(nodo i,S,Cq)										O(n*log(|Cq|))	        
     para todo w vecino de i 
        si(w no esta en Cq y no fue encolado en S)
          encolar w en S
fin

expandirClique(Cq,S)								
    mientras S no vacía					 			
        w=tope de S									
        Si esClique(w,Cq)       O(|Cq|)				
            insertar w en Cq    O(log(|Cq|))  } * O(n) = O(n*n*log(|Cq|)) = O(n*n*log(n)))
            merge(w,S,Cq)       O(n*log(|Cq|))		
        desencolar w de S		              
fin												
									
HC()										 	
    generarDensidad()           O(n*n)		   	
    insertar nodoMax en Cq      O(1)            } = O (n*n*log(n)) 	
    merge(NodoMax,S,Cq)         O(n)			   
    expandirClique(Cq,S)        O(n*n*log(n))	 	
fin												

\end{verbatim}
La función merge itera en peor caso sobre n-1 vecinos y para cada uno pregunta si está en Cq ,lo cual tiene un costo de log(|Cq|) según referencia de la función count de set de STL. 
Podemos observar que la complejidad de la función expandirClique determina que el peor caso para el algoritmo HC es el grafo completo, ya que la misma itera sobre el conjunto S de los candidatos sin repetir hasta que éste quede vacío; en cada iteración, Cq aumenta en uno (w) y se mergean los vecinos de w a S.
La complejidad nos queda expresada así ,donde i es el tamaño de Cq que se va incrementando en cada interación y los 3 terminos de la sumatoria corresponden a las funciones esClique,insertar y merge dentro del ciclo

$\displaystyle \sum_{i=1}^{n} i + log(i) + n*log(i) <= \frac{n(n-1)}{2}+\sum_{i=1}^{n} (n+1)*log(i) <=$  
 
$\displaystyle  <= \frac{n(n-1)}{2}+(n+1)*log(i!) <= \frac{n(n-1)}{2}+(n+1)*n*log(n) $

Luego la complejidad de HC es O(n*n) + O(1) + O(n) + O(n*n*log(n)) = O(n*n*log(n))  

\textbf{Heurística Local}

La heurística Local se define como un algoritmo que se mueve de clique maximal en clique maximal. Recibe al iniciar una clique maximal, luego establece un como vecindad a un conjunto de nodos para reemplazar a algun vértice del clique, sin cambiar su tamaño. Estos nuevos nodos tienen la propiedad de estar conectados con todos menos algun vértice del clique. En caso de que exista algun nodo en esta vecindad que aumente la suma total de $delta$ en el clique, se lo intercambia por su "opuesto". Donde el opuesto se refiere al nodo del clique no conectado con este candidato. A este último paso lo llamamos "desplazamiento". Luego de cada desplazamiento exitoso, esto significa haber reemplazado algun nodo, se intenta expandir. Si no se pudo desplazar no tiene sentido expandir, ya que el grafo es maximal, mientras que si no pudimos ni desplazar ni expandir el algoritmo terminó. A continuación el pseudocódico de las funciones principales.


\begin{verbatim}


set vecindad(clique) {
    multiset vecindad
    set resultado
    //de todos los posibles nodos vecinos.
    para cada c en clique				// O(n)
		para cada v en vecinos(c)		// O(n)
			vecindad.agregar(v)			// O(log(n))
		fin para
	fin para
	para cada n de vecindad				// O(n)
		si multiset.cantidad(n) == clique.tamaño	// O(n*log(n))
			resultado.insert(n)			// O(log(n))
		fin si
	fin para
	return resultado
}


findCandidato(clique) {
	candidato res;
    res.delta = 0;
    posibles = vecindad(cq); 			//O(n*n*log(n)*log(n))
    para cada v en posibles :			//O(n)
        opuesto = findOpuesto(v, clique);	//O(n)
        // op es el opuesto a v
        si delta(v) - delta(opuesto) > res.delta
			// poner a v es mejor que dejar a opuesto. y mejor que cualquier anterior (res.delta >= 0)
            res.nuevo = v;
            res.viejo = opuesto;
			res.delta = delta(v) - delta(opuesto);
        fin si
    fin para
    return res;

}

HL() {
	clique = HC();	// O(n*n*log(n))
	candidato = findCandidato(clique) // O(n*n*log(n)*log(n))
	mientras candidato.delta > 0		//O(n)
		clique.erase(candidato.viejo)	//O(log(n))
		clique.add(candidato.nuevo)				//O(log(n))
		pqDelta S; 		//cola de prioridad ordenada por delta
		merge(clique.dameUno(),S,clique);	//O(n*n*log(n))
		expandClique(clique, S)				//O(n*n*log(n))
		candidato = findCandidato(clique)	//O(n*n*log(n)*log(n))			
	fin mientras
    return clique;
}
\end{verbatim}

Recordamos que son cotas bastantes groseras, son a modo ilustrativo para ver que la heurística tiene complejidad polinomial. Para cada iteración que itera sobre nodos se considera que toma instrucciones proporcionales a la cantidad de nodos, cuando la mayoría de las veces es en proporción al tamaño de la clique. Las complejidades de los TADs como conjuntos o multiconjuntos son las complejidades de la implementación de tales tipos en la libreria estandard de C++ \footnote{http://www.cplusplus.com/}.
Utilizando algebra de Ordenes podemos llegar facilmente a que la complejidad del algoritmo (HL) para el peor caso es $O(n^4 \times log(n)) $.

En las pruebas que realizamos no pudimos encontrar ninguna vez en que mejorara a la Heuristica Constructiva. De todas formas creemos que los grafos para los que dará soluciones malas seguramente serán muy parecidos a los grafos en los cuales la Heurística Constructiva también le ocurra.

\subsubsection{Variantes}

\textbf{Heurística Local}

Como variantes al algoritmo propuesto también surgió la idea de buscar en la vecindad, no solo nodos con  $k-1$ vecinos en común con el clique (clique de tamaño $k$). Sino también cliques entre los nodos vecinos a la clique que tuvieran $k-l$ vecinos en comun con el clique (otra vez $l$ tamaño de este nuevo clique. Pero algoritimicamente esto nos resultó muy dificultoso.
\\
Por otro lado también surgió la idea de utilizar un clique maximal inicial distinto al entregado por la Heurística Constructiva. Pero esto representaba implementar todo un nuevo algoritmo para hallar cliques con algun criterio distinto.

