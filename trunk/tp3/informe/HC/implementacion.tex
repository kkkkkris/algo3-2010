El ejercicio compila con el comando \texttt{make} .\\
El algoritmo recibe como parámetros de entrada el modo (HC para heurística constructiva y HL para heurística local) y el nombre de archivo \emph{ñ\'nombre'} se ejecuta \texttt{./max\_clique modo entrada.in}.\\
El archivo \emph{'entrada'} (.in) contiene una lista de grafos y el fin de la lista lo determina una línea con -1.\\
El algoritmo genera el archivo de salida \emph{'nombre+modo'} (.out)con los resultados del max\_clique hallado para cada grafo.\\
Por último el archivo \emph{'nombre+modo'} (.times)  también generado por el algoritmo, contendrá una tabla con el tamaño de cada grafo (n) y tiempo de ejecución correspondiente, para cada uno.\\
Los grafos para la experimentación se tomaron de \emph{http://cs.hbg.psu.edu/benchmarks/clique.html}, además se corrieron grafos completos que fueron generados a efectos de graficar el peor caso y se construyeron con con un generador de grafos completos.