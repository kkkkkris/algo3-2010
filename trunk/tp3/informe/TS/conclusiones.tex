Viendo los resultados, podemos ver que las funciones objetivos resultan bastante exitosas, llegando a encontrar las cliques máximas para ciertos grafos. Vemos que para ciertos casos conviene efectuar $\delta^{1}$ o $\delta^{2}$. Particularmente la metaheurística resulta bastante satisfactoria considerando que no hemos introducido demasiados parámetros de ajuste, y que ha dado buenos resultados para tipos de grafos variados. Cabe destacar que podrían esperarse resultados mejores (como los obtienen Patrick y Gendreau en su trabajo) si se introducen funciones objetivo más refinadas, o mecanismos como el uso de más de una lista tabú de largo plazo, o aspectos probabilisticos que permitan explorar de forma mas aleatoria el grafo con la esperanza de encontrar soluciones mejores.