Se adjunta con el trabajo un script bash que se usó para correr los tests llamado \textit{test.sh}, junto con un \textit{Tp3.in} que contiene los casos de prueba usados.\\
El programa no solo informa en Tp3TS.out el resultado con el formato indicado, sino que por consola informa el tamaño del clique máximo, n, m y el tiempo más corto de 5 corridas iguales (en nanosegundos).
Para la entrega esta porción de código está comentada, pero si se quieren repetir los tests, descomentar esa parte, recompilar el tp, y correr test.sh (que guarda todo lo mostrado por consola en 
el archivo tests_resultados).